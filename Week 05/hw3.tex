% Created 2016-02-07 Sun 19:53
\documentclass[11pt]{article}
\usepackage[utf8]{inputenc}
\usepackage[T1]{fontenc}
\usepackage{fixltx2e}
\usepackage{graphicx}
\usepackage{longtable}
\usepackage{float}
\usepackage{wrapfig}
\usepackage{rotating}
\usepackage[normalem]{ulem}
\usepackage{amsmath}
\usepackage{textcomp}
\usepackage{marvosym}
\usepackage{wasysym}
\usepackage{amssymb}
\usepackage{hyperref}
\tolerance=1000
\usepackage{minted}
\usemintedstyle{tango}
\author{}
\date{\today}
\title{Breaking and fixing the Java Memory Model for profit}
\hypersetup{
  pdfkeywords={},
  pdfsubject={},
  pdfcreator={Emacs 24.4.1 (Org mode 8.2.10)}}
\begin{document}

\maketitle
\tableofcontents


\section{Answers and Summary}
\label{sec-1}
For the benefit of the T.A.s, I summarize my results and
findings for this homework before any of the material is
presented. This is because I have included an extensive
write up of the homework. Thus, this summary will attempt
to meet the requirements of the homework specification,
while the details for these findings are found in the
pages that follow.

The \verb~Synchronized~ and \verb~Null~ models provide a baseline
to begin our analysis. Looking at the results in the
following pages, we can see that the \verb~Null~ model runs
around two to three times faster than \verb~Synchronized~.
Thus, if our class implementations are done right, we
want to find ourselves somewhere in between these two
models. The Unsynchronized model, when it does not deadlock
should run at about the same speed as \verb~Null~; however,
I was never able to get it to complete the entire test
suite because of lockups. Hence, it results appear to
be significantly slower. But this is only because the
number of swaps is significantly smaller than the ones
we've done on the previous two models.

We avoid deadlocks and erronous ouput using the \verb~AtomicIntegerArray~
API that Java provides for us for our \verb~GetNSet~ model.  As the
specification indicates, it runs at a speed that is in between \verb~Null~
and \verb~Synchronized~. This is probably due to overhead that
\verb~Synchronized~ imposes on the execution of a method. Thus, this is a
desirable outcome for our company GDI. The implementation and
execution of \verb~BetterSafe~ is similarly faster than \verb~Synchronized~.
In fact, \verb~BetterSafe~ runs almost 2.5x faster than \verb~Synchronized~.
This model uses Java's Reentrant lock API. Since its overhead appears
to be much smaller than \verb~Synchronized~, it achieves faster over
run time. Still, with this speed up, it is able to maintain 
error-free results.

Finally, if our company GDI wants speed at the expense of a little
accuracy, the best model is \verb~BetterSorry~ as it achieves speeds
as fast as \verb~Null~, while sacrificing a little bit of accuracy.
I achieved this by creating a \verb~static volatile~ boolean that gets
flipped on and off when the program is executing critical portions
of code, i.e., writing to memory. It still contains race conditions
during reads, though these have been minimized to a maximum of two
instructions. A program that increases the liklihood of race conditions
is one that invokes a higher and higher number of threads.

\section{Introduction}
\label{sec-2}
\subsection{Checking environment variables and Java version}
\label{sec-2-1}
\subsubsection{Java version}
\label{sec-2-1-1}
Before we get started doing anything else in the assignment,
let's make sure that we have our environment variables and
an appropriate version of Java. I'll be using a local copy
of JDK version 1.8.071:

\begin{minted}[]{sh}
java -version
\end{minted}
Java(TM) SE Runtime Environment (build 1.8.0$_{\text{71}}$-b15)
Java HotSpot(TM) 64-Bit Server VM (build 25.71-b15, mixed mode)
\subsubsection{CPU information}
\label{sec-2-1-2}
We can look at our CPU information using \verb~less~ to see that we will
be using an Inter Core i3 Processor running at 3.3GHz. My particular
machine reports a single core. This may or may not affect the way
interleaved CPU instructions are executed on several cores. Thus,
I will run my code locally and on UCLA's SEASNET servers. 

\begin{minted}[]{sh}
less /proc/cpuinfo
\end{minted}
\subsubsection{Memory Information}
\label{sec-2-1-3}
We can similarly inspect our machine's memory information by look at
the `/proc/meminfo' file, which tells me that I have 764300 kB of
memory total.
\section{Running the tests}
\label{sec-3}
\subsection{Extracting jmm.jar}
\label{sec-3-1}
The files that we'll need for this assignment are compressed in 
the `jmm.jar' file. Thus, we'll need to decompress the jar file
before we can do anything else:

\begin{minted}[]{sh}
jar -xvf executables/jmm.jar
\end{minted}
created:   META-INF/              
inflated:  META-INF/MANIFEST.MF   
inflated:  NullState.java         
inflated:  State.java             
inflated:  SwapTest.java          
inflated:  SynchronizedState.java 
inflated:  UnsafeMemory.java      
\subsection{Makefile}
\label{sec-3-2}
Compiling the source files into executable class files will become tedious.
We can automate this process using a Makefile:

\begin{minted}[]{make}
JC = javac
OUTPUT = executables/

all: nullstate swaptest synchronized_state state unsafe_memory

# All classes for this assingnment
nullstate : NullState.java
	$(JC) -d $(OUTPUT) NullState.java
swaptest : SwapTest.java
	$(JC) -d $(OUTPUT) SwapTest.java
synchronized_state : SynchronizedState.java
	$(JC) -d $(OUTPUT) SynchronizedState.java
state: State.java
	$(JC) -d $(OUTPUT) State.java
unsafe_memory: UnsafeMemory.java
	$(JC) -d $(OUTPUT) UnsafeMemory.java

clean:
	rm $(OUTPUT)*
\end{minted}
\subsection{Testing}
\label{sec-3-3}
With the prerequisite files and environment variables all in order, we
can begin testing and working on the assignment. We'll begin our tests
on the Synchronized implementation, moving over to the Null model
afterwards.
\subsubsection{Synchronized model}
\label{sec-3-3-1}
Here we test the synchronized model first. The initial results on my
local machine indicate that this program is not benefiting from more
threads. In fact, the more cores we add, the worse our program
performs. Given, these results, I will move over to the SEASNET
servers to test application peformance on a multicore machine.

\begin{verbatim}
Threads average 69.9444 ns/transition
Threads average 513.126 ns/transition
Threads average 1511.14 ns/transition
Threads average 3254.82 ns/transition
Threads average 6098.57 ns/transition
Threads average 12087.3 ns/transition
\end{verbatim}

Below are the tests and results from running the same
application on the SEASNET servers. The results from
the SEASNET servers run approximately twice as fast
as the results my local machine produced. 

\begin{verbatim}
Synchronized tests
first test set
01: Threads average 70.2679 ns/transition
02: Threads average 425.178 ns/transition
04: Threads average 1637.13 ns/transition
08: Threads average 2989.45 ns/transition
16: Threads average 5828.51 ns/transition
32: Threads average 14677.7 ns/transition
\end{verbatim}

\begin{verbatim}
second test set
01: Threads average 108.989 ns/transition
02: Threads average 740.129 ns/transition
04: Threads average 2763.78 ns/transition
08: Threads average 4828.20 ns/transition
16: Threads average 7848.66 ns/transition
32: Threads average 21089.5 ns/transition
\end{verbatim}

\begin{verbatim}
thirds test set
01: Threads average 90.7319 ns/transition
02: Threads average 530.770 ns/transition
04: Threads average 2188.73 ns/transition
08: Threads average 4002.67 ns/transition
16: Threads average 8794.03 ns/transition
32: Threads average 18009.1 ns/transition
\end{verbatim}


\subsubsection{Null model}
\label{sec-3-3-2}
As indicated by the specification for this assignment, the
Null model does not yet work but still passes the test,
thus it runs to completion much faster than the synchronized
model. We should note the overhead of creating threads
at least on this local machine adds considerable running
time to our program despite the fact that no actual work
is being done.

\begin{verbatim}
01 Threads average 36.7658 ns/transition
02 Threads average 132.399 ns/transition
04 Threads average 455.818 ns/transition
08 Threads average 2336.21 ns/transition
16 Threads average 4654.78 ns/transition
32 Threads average 8696.07 ns/transition
\end{verbatim}

\begin{minted}[]{sh}
cd files/executables;
echo "second test set";
echo -n "01: "; java UnsafeMemory Null 1 1000000 2 1 1 0 0 1
echo -n "02: "; java UnsafeMemory Null 2 1000000 2 1 1 0 0 1
echo -n "04: "; java UnsafeMemory Null 4 1000000 2 1 1 0 0 1
echo -n "08: "; java UnsafeMemory Null 8 1000000 2 1 1 0 0 1
echo -n "16: "; java UnsafeMemory Null 16 1000000 2 1 1 0 0 1
echo -n "32: "; java UnsafeMemory Null 32 1000000 2 1 1 0 0 1
\end{minted}


\begin{verbatim}
thirds test set
01: Threads average 36.5781 ns/transition
02: Threads average 138.304 ns/transition
04: Threads average 439.491 ns/transition
08: Threads average 2372.14 ns/transition
16: Threads average 4792.37 ns/transition
32: Threads average 8229.13 ns/transition
\end{verbatim}

\section{Unsynchronized implementation}
\label{sec-4}
We can begin implementing the unsynchronized model by bringing over
the code from the synchronized model and tinkering with it. We will
start with a basic class definition, naming the class
UnsynchronizedState and letting the Java compiler know that we'll be
implementing the class State. This means we'll have to take all the
method signatures from State and actually implement them here:

\begin{minted}[]{java}
class UnsynchronizedState implements State {
    private byte[] value;
    private byte maxval;
\end{minted}

Similar to the synchronized version, we'll have two constructors: a
constructor that receives an array to initialize to some value, and
sets the maximum value for the object to 127. We also have a second
constructor that similarly takes in an array but also takes in a
byte, setting the maximum value for this object to m.

\begin{minted}[]{java}
UnsynchronizedState(byte[] v) { value = v; maxval = 127; }

UnsynchronizedState(byte[] v, byte m) { value = v; maxval = m; }
\end{minted}

The key change to the class is simply a removal of the keyword
\verb~synchronized~ from the definition of the swap method:

\begin{minted}[]{java}
    public int size() { return value.length; }

    public byte[] current() { return value; }

    public boolean swap(int i, int j) {
	if (value[i] <= 0 || value[j] >= maxval) {
	    return false;
	}
	value[i]--;
	value[j]++;
	return true;
    }
}
\end{minted}

We can compile our class and test it like the other two we've
tested before:

\begin{minted}[]{sh}
cd files;
make unsynchronized_state
\end{minted}

Finally, before we can run our program again, we need to ensure that
our program knows how to use the new class by adding two lines of code:
\begin{minted}[]{java}
else if (args[0].equals("Unsynchronized"))
    s = new UnsynchronizedState(stateArg, maxval);
\end{minted}

\subsection{Running Unsynchronized}
\label{sec-4-1}
There is a problem with the way that unsynchronized works. When we
increase the number of threads or swaps beyond an arbitrary value
the likelihood that the program will become deadlocked increases.
Thus, for these tests we used orders of magnitude smaller swaps
than previous tests:

01 Threads average 3412.14 ns/transition
02 Threads average 6278.02 ns/transition
sum mismatch (17 != 12)
04 Threads average 13747.9 ns/transition
sum mismatch (17 != 18)
08 Threads average 26983.3 ns/transition
sum mismatch (17 != 19)
16 Threads average 56221.0 ns/transition
sum mismatch (17 != 16)
32 Threads average 152139 ns/transition
sum mismatch (17 != 16)

As expected, our unsynchronized class runs into race conditions, where we
get unexpected unreliable values.

\section{GetNSet}
\label{sec-5}
\subsection{Writing the Class}
\label{sec-5-1}
With the problematic \emph{unsynchronized} class implemented, we want
to achieve similar speed but without the race conditions. Is that
possible? Lets implement Java's atomic integer array and see if
we can do any better. A definition provided on Wikipedia states
that an atomic operation is one that is a guarantee of isolation
from concurrent processes. Since we'll be using the
AtomicIntegerArray class, lets include it in our file and
declare a variable \verb~valueIntegerArray~ that we'll instantiate
in our constructor:

\begin{minted}[]{java}
import java.util.concurrent.atomic.AtomicIntegerArray;

class GetNSet implements State {
    private int[] value;
    private byte maxval;
    private AtomicIntegerArray valueIntegerArray;
\end{minted}

With the variable declared above, we'd like to instantiate
an instance of the class; however, looking at the documentation
for AtomicIntegerArray shows us that we need to pass in an
integer array, not a byte array. Thus, we'll want to repurpose
\verb~value~ as an \verb~int~ array and run a loop that will set each
element its equivalent in the byte array:

\begin{minted}[]{java}
GetNSet(byte[] v) {
    value = new int[v.length];

    for(int i = 0; i < value.length; i++){
	value[i] = v[i];
    }

    maxval = 127;
    valueIntegerArray = new AtomicIntegerArray(value);
}

GetNSet(byte[] v, byte m) { 
    value = new int[v.length];

    for(int i = 0; i < value.length; i++){
	value[i] = v[i];
    }

    maxval = m;
    valueIntegerArray = new AtomicIntegerArray(value);
}
\end{minted}

With the constructors that correctly instantiate our AtomIntegerArray
we can change the size method so that it gets the AtomicIntegerArray
length. We just call its \verb~length~ method. The \verb~current~ method requires
us to return a \verb~byte~ array, so we'll need to create a temporary
byte array and return it:

\begin{minted}[]{java}
public int size() { return valueIntegerArray.length(); }

public byte[] current() {
    byte[] tmp = new byte[value.length];

    for(int i = 0; i < tmp.length; i++){
	tmp[i] = (byte) value[i];
    }

    return tmp;
}
\end{minted}

Finally, the \verb~swap~ function needs to use the \verb~get~ and \verb~set~ methods
provided by the AtomicIntegerArray class:

\begin{minted}[]{java}
    public boolean swap(int i, int j) {
	if (valueIntegerArray.get(i) <= 0 || valueIntegerArray.get(j) >= maxval) {
	    return false;
	}
	valueIntegerArray.getAndDecrement(i);
	valueIntegerArray.getAndIncrement(j);
	return true;
    }
}
\end{minted}
\subsection{Results}
\label{sec-5-2}
Let's run this class, the same way we've done before:

\begin{minted}[]{sh}
cd files/executables;
echo -n "01 "; java UnsafeMemory GetNSet 1 1000000 6 5 6 3 0 3
echo -n "02 "; java UnsafeMemory GetNSet 2 1000000 6 5 6 3 0 3
echo -n "04 "; java UnsafeMemory GetNSet 4 1000000 6 5 6 3 0 3
echo -n "08 "; java UnsafeMemory GetNSet 8 1000000 6 5 6 3 0 3
echo -n "16 "; java UnsafeMemory GetNSet 16 1000000 6 5 6 3 0 3
echo -n "32 "; java UnsafeMemory GetNSet 32 1000000 6 5 6 3 0 3
\end{minted}

Like our previous results, we'd expect that the more threads we add
the faster our program should run; however, it looks like the overhead
of creating the threads is too costly for this simple swap function.
On a positive note, we are no longer getting bad results, even testing
on an array two and three orders of magnitude larger produces no
bad results:

\begin{center}
\begin{tabular}{rllrl}
32 & Threads & average & 3621.75 & ns/transition\\
32 & Threads & average & 3904.25 & ns/transition\\
\end{tabular}
\end{center}

\section{BetterSafe}
\label{sec-6}
\subsection{Writing the class}
\label{sec-6-1}
We can now move to the BetterSafe model, which will
achieve better performance than \emph{Synchronized} but
still maintain 100\% reliability. We will be able to
do this by implementing a system of locks and unlocks.

We begin with our familiar code from \emph{Synchronized},
maintaining a majority of the code. Thus, we only
change the name of the class along with the constructor
names to reflect this change. Finally, we'll add a
lock to use when we are performing a swap:

\begin{minted}[]{java}
import java.util.concurrent.locks.ReentrantLock;

class BetterSafe implements State {
    private byte[] value;
    private byte maxval;
    private final ReentrantLock swapLock;

    BetterSafe(byte[] v) {
	value = v; maxval = 127;
	swapLock = new ReentrantLock();
    }

    BetterSafe(byte[] v, byte m) {
	value = v; maxval = m;
	swapLock = new ReentrantLock();
    }
\end{minted}

We'll remove the \verb~synchronized~ keyword from the swap
function and implement a use of locks to make sure that
no thread steps on anyone else's toes: 

\begin{minted}[]{java}
public int size() { return value.length; }

public byte[] current() { return value; }

public boolean swap(int i, int j) {
    swapLock.lock();

    if (value[i] <= 0 || value[j] >= maxval) {
	swapLock.unlock();

	return false;
    }
    value[i]--;
    value[j]++;

    swapLock.unlock();

    return true;
			    }
}
\end{minted}
\subsection{Testing BetterSafe}
\label{sec-6-2}
Let's test our BetterSafe class by performing the same tests
that we've done in the past:

\begin{center}
\begin{tabular}{rllrl}
orginal & test: &  &  & \\
1 & Threads & average & 78.0822 & ns/transition\\
2 & Threads & average & 549.396 & ns/transition\\
4 & Threads & average & 624.036 & ns/transition\\
8 & Threads & average & 1160.22 & ns/transition\\
16 & Threads & average & 2405.32 & ns/transition\\
32 & Threads & average & 5874.64 & ns/transition\\
larger & test: &  &  & \\
1 & Threads & average & 78.6375 & ns/transition\\
2 & Threads & average & 582.671 & ns/transition\\
4 & Threads & average & 562.409 & ns/transition\\
8 & Threads & average & 1207.74 & ns/transition\\
16 & Threads & average & 2455.95 & ns/transition\\
32 & Threads & average & 5522.98 & ns/transition\\
\end{tabular}
\end{center}

\section{BetterSorry}
\label{sec-7}
\subsection{Writing BetterSorry}
\label{sec-7-1}
\begin{minted}[]{java}
import java.util.concurrent.TimeUnit;

class BetterSorry implements State {
    private volatile byte[] value;
    private byte maxval;
    private static volatile boolean inCritical = false;
\end{minted}

Similar to the synchronized version, we'll have two constructors: a
constructor that receives an array to initialize to some value, and
sets the maximum value for the object to 127. We also have a second
constructor that similarly takes in an array but also takes in a
byte, setting the maximum value for this object to m. We'll use a
psuedo-lock by creating a boolean that lets us know when we're in
a critical part of the execution, i.e., when we're writing to our
array.

\begin{minted}[]{java}
BetterSorry(byte[] v) { value = v; maxval = 127; }

BetterSorry(byte[] v, byte m) { value = v; maxval = m; }
\end{minted}

To make sure we don't have any deadlocks, we'll check to make
sure we are not in a critical section, i.e., writing to
our array. If we are, we'll wait our turn. If not, then the
thread will write what it needs to the array.

\begin{minted}[]{java}
    public int size() { return value.length; }

    public byte[] current() { return value; }

    public boolean swap(int i, int j) {
	int v_i = value[i], v_j = value[j];

	if (v_i <= 0 || v_j >= maxval) {
	    return false;
	}
	while(inCritical) {
	    try {
		TimeUnit.NANOSECONDS.sleep(1);
	    } catch (InterruptedException e) {
		// TODO Auto-generated catch block
		e.printStackTrace();
	    }
	}

	inCritical = true;

	value[i]--;
	value[j]++;

	inCritical = false;


	return true;
    }
}
\end{minted}
\subsection{Testing BetterSorry}
\label{sec-7-2}

\begin{center}
\begin{tabular}{rllrl}
01 & Threads & average & 69.8024 & ns/transition\\
02 & Threads & average & 141.296 & ns/transition\\
04 & Threads & average & 295.705 & ns/transition\\
08 & Threads & average & 644.861 & ns/transition\\
sum mismatch (17 != 18) &  &  &  & \\
16 & Threads & average & 1722.14 & ns/transition\\
sum mismatch (17 != 23) &  &  &  & \\
32 & Threads & average & 4318.05 & ns/transition\\
sum mismatch (17 != 24) &  &  &  & \\
\end{tabular}
\end{center}
% Emacs 24.4.1 (Org mode 8.2.10)
\end{document}
